\section{图像分块与电平平移归一化}

在图像变换之前,首先需要对图像进行预处理。预处理主要分为分块、电平平移归一化和颜色变换三个过程(\textbf{图\ref{fig1}})。在调用\textit{cv2.imread}读取图像数据时,利用正则表达式获取图像的颜色位数信息,方便之后的处理。\textit{代码见函数init\_image。}

\begin{figure}[H]
	\centering
	\includegraphics[scale=0.4]{fig1}
	\caption{图像预处理.}
	\label{fig1}
\end{figure}

\subsection{图像分块}
与JPEG标准相同,JPEG2000标准也需要将图像分块,不过是为了降低计算量和对内存的需求,以方便压缩(当然也可以选择不分块)。但与JPEG标准不同的是,JPEG2000标准并不需要将图像强制分成$8\times 8$的小块,而是可以将图像分割成若干任意大小的互不重叠的矩形块(tile),常分为$2 ^6\sim 2^{12}$(即$64\sim 1024$像素宽)的正方形tile。受图像形状的影响,边缘部分的tile的大小和形状可能与其它的tile不同。Tile的大小会影响重构图像的质量。一般情况下,大的tile重构出的图像质量要好一些,因为小波变换的范围大,减轻了边缘效应。\textit{代码见class Tile和class JPEG2000中的函数image\_tiling。} \par

\subsection{电平平移归一化}
电平平移归一化分为两步:\par
第一步是直流电平平移(DC level shifting),通过将数据减去均值使之关于0对称分布,以去掉频谱中的直流分量。
\[
\begin{aligned}
I_1(x,y)=I(x,y)-2^{B-1}\\
-2^{B-1}\le I_1(x,y)\le 2^{B-1}-1
\end{aligned}
\]
其中B是之前读取到的颜色位数信息。\par
第二步是电平归一化(normaliztion)。对于无损压缩中的5/3小波变换,由于采用的是整数小波变换,所以不需要进行归一化;而对于有损压缩中的9/7小波变换,由于采用的是实数小波变换,故需要对每个tile进行如下运算以归一化:\par
\[
\begin{aligned}
I_2(x,y)=\frac{I_1(x,y)}{2^{B}}\\
-\frac{1}{2}\le I_2(x,y)<\frac{1}{2}
\end{aligned}
\]
但是事实上,电平平移归一化对DWT来说不是必须的(而且电平平移归一化后还不便于进行色彩空间变换),因为它只会影响小波系数的动态范围而不会影响结果。\textit{代码见函数dc\_level\_shift。}

\subsection{反变换}
在解码、小波反变换和色彩空间反变换之后,需要进行电平反平移,(如果是有损压缩,还需要进行反归一化),最后拼接得到恢复的图像。这些都是上述操作的反过程,这里不再赘述。\textit{代码见函数idc\_level\_shift和image\_splicing。}



